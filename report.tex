% version: php developer

\documentclass[letterpaper,12pt]{article}
\usepackage{latexsym}
\usepackage[empty]{fullpage}
\usepackage{titlesec}
\usepackage{marvosym}
\usepackage[usenames,dvipsnames]{color}
\usepackage{verbatim}
\usepackage{enumitem}
\usepackage[hidelinks]{hyperref}
\usepackage{fancyhdr}
\usepackage[T2A]{fontenc}
\usepackage[utf8]{inputenc}
\usepackage{amsmath,amssymb}
\usepackage[russian,english]{babel}
\usepackage{tabularx}
\input{glyphtounicode}

\usepackage{helvet}
\usepackage{geometry}
\geometry{papersize={8.27in, 12.6in}}

\pagestyle{fancy}
\fancyhf{} % Clear all header and footer fields
\fancyfoot{}
\renewcommand{\headrulewidth}{0pt}
\renewcommand{\footrulewidth}{0pt}

% Adjust margins
\addtolength{\oddsidemargin}{-0.5in}
\addtolength{\evensidemargin}{-0.5in}
\addtolength{\textwidth}{1in}
\addtolength{\topmargin}{-.5in}
\addtolength{\textheight}{1.0in}

\urlstyle{same}

\raggedbottom
\raggedright
\setlength{\tabcolsep}{0in}

% Sections formatting
\titleformat{\section}{
  \vspace{-4pt}\scshape\raggedright\large
}{}{0em}{}[\color{black}\titlerule \vspace{-5pt}]

% Ensure that generate PDF is machine readable/ATS parsable
\pdfgentounicode=1

%-------------------------
% Custom commands
\newcommand{\resumeItem}[2]{
  \item\small{
    \textbf{#1}{: #2 \vspace{-2pt}}
  }
}

% Just in case someone needs a heading that does not need to be in a list
\newcommand{\resumeHeading}[4]{
    \begin{tabular*}{0.99\textwidth}[t]{l@{\extracolsep{\fill}}r}
      \textbf{#1} & #2 \\
      \textit{\small#3} & \textit{\small #4} \\
    \end{tabular*}\vspace{-5pt}
}

\newcommand{\resumeSubheading}[4]{
  \vspace{-1pt}\item
    \begin{tabular*}{0.97\textwidth}[t]{l@{\extracolsep{\fill}}r}
      \textbf{#1} & #2 \\
      \textit{\small#3} & \textit{\small #4} \\
    \end{tabular*}\vspace{-5pt}
}

\newcommand{\resumeSubSubheading}[2]{
    \begin{tabular*}{0.97\textwidth}{l@{\extracolsep{\fill}}r}
      \textit{\small#1} & \textit{\small #2} \\
    \end{tabular*}\vspace{-5pt}
}

\newcommand{\resumeSubItem}[2]{\resumeItem{#1}{#2}\vspace{-4pt}}

\renewcommand{\labelitemii}{$\circ$}

\newcommand{\resumeSubHeadingListStart}{\begin{itemize}[leftmargin=*]}
\newcommand{\resumeSubHeadingListEnd}{\end{itemize}}
\newcommand{\resumeItemListStart}{\begin{itemize}}
\newcommand{\resumeItemListEnd}{\end{itemize}\vspace{-5pt}}
\usepackage{listings}
%-------------------------------------------
%%%%%%  CV STARTS HERE  %%%%%%%%%%%%%%%%%%%%%%%%%%%%

\begin{document}

\begin{center}
    
    \textbf{\Huge Выполнил Бабенко Роман}

  \end{center}
  
  Код для полинома лагранжа был выполнен и показан 
  прямо на паре, поэтому в этом отчёте не будет 
  его объяснения. Но его и все другие программы 
  можно найти в приложении к отчёту.

  также всё выложено на github: \href{https://github.com/skrabik/hw/}{https://github.com/skrabik/hw/}

  \textbf{Задача 1.}
 \\~\\
  Нужно найти оченку погрешности:
    \\~\\
   \begin{center} $|f(x)-L_n(x)| \leq \frac{M_n|w_n(x)|}{n!}$
    
   \end{center}


    Найдём сначала значение $M_n$.
    С помощью sympy найдём четвёртую производную
    от $\frac{1}{x}$ и $M_n$
    с помошью такого кода: 
    \begin{lstlisting}
        x = sympy.Symbol('x')
        f = 1/x
        fourth_derivative = sympy.diff(f, x, 4)

        def four_f(x):
            return 24 / (x**5)

        Mn = max([four_f(x) for x in Xxx])
    \end{lstlisting}

    Далее функцию $w_n$ и функцию выкториала

    \begin{center}
        $w_n(x) = \prod_{j=1}^{n}(x-x_{j})$
    \end{center}

    \begin{lstlisting}
        def wk (x):
            res = 1
            for j in range(4):
                res *= (x - Xi[j])
            return res

        def fact(n):
            if n == 1 or n == 0:
                return 1
            return fact(n-1)*n

        res = list()
    \end{lstlisting}

    Теперь просерим утверждение. Пройдёмся по 
    всем x, и запишем значение выражения в 
    каждой точке.
    \begin{lstlisting}
    for x in Xxx:
        expression = ((abs(f(x) - float(lag_pol(x)))) <= (Mn 
        * abs(wk(x) / fact(4))))
        if not expression:
            print(x)
        res.append(expression)
    print(res.count(True), res.count(False))
    \end{lstlisting}

    Далее смотрим количестов значаний "True" и "False":
    800 0. Наше выражение верно!

    Аналогичный результат получаем для 
    многочленов лагранжа, посроенных на 3-х и 2-х
    точках. (Весть подробный код есть в ipyng файле)
    \clearpage
    Производные для вычисления $M_n$ получатся:

    \begin{lstlisting}
        
        def three_f(x):
            return -6 / (x**4)

        def two_f(x):
            return 2 / (x**3)

    \end{lstlisting}

    \textbf{Задача 2}
    \\~\\
    Необходимо найти оценку погрешности с 
    помощью разделённых разностей.

    \begin{center}

        $f(x) - L_n(x) = f(x_1; x_2; ... 
        ;x_n) w_n(x)$

    \end{center}

    Для вычисления $w_n$ мы воспользуемся 
    уже написанной функцией из прошлой задачи.
    \\
    Для фукции вычислления раздёлённой разности 
    я написал такую рекурсивную функцию:

    \begin{lstlisting}
        def f_razd (xvalues = list()):
            if len(xvalues) == 2:
                return ((f(xvalues[1]) - f(xvalues[0]) ) / (xvalues[1] -  xvalues[0])) 
            return f_razd(xvalues[1:]) - f_razd(xvalues[:-1]) /
             (xvalues[-1] / xvalues[0])
    \end{lstlisting}

    Соответственно также пробегаемся по всем x,
    и проверяем значение выражения:

    \begin{lstlisting}
        res = list()
        for x in Xxx:
            expression = ((f(x) - lag_pol(x)) == (f_razd(Xi)*wn(x)))
            res.append(expression)

        print(res.count(True), res.count(False))
    \end{lstlisting}

    Мы получили ответ: 800 0!

    \begin{flushright}
        ps. подробные программы можно 
        найти в прилагаемом файле lagrange.ipynb
    \end{flushright}

\end{document}